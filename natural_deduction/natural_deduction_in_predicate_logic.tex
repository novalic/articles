\documentclass{article}
\usepackage[utf8]{inputenc}
\usepackage{graphicx}
\usepackage{bussproofs}
\usepackage{amssymb}
\usepackage{hyperref}
\usepackage[a4paper, total={7in, 10in}]{geometry}

\newenvironment{scprooftree}[1]%
  {\gdef\scalefactor{#1}\begin{center}\proofSkipAmount \leavevmode}%
  {\scalebox{\scalefactor}{\DisplayProof}\proofSkipAmount \end{center} }

\begin{document}

{\Huge Natural Deduction on Predicate Logic} \\

All this work is based on the book \textit{Logic and Structure} by Dirk Van Dalen.

\section*{Strategy to build correct derivations in Predicate Logic}

\noindent\fbox{%
    \parbox{\textwidth}{%
\textbf{From an university test:} \\ \\

Let $\Gamma = \{(\exists x)P_1(x), (\exists x)(\exists y)(\neg P_2 (x, y)), (\forall x)(P_1 (x) \rightarrow (\exists y)P_2 (x,y)) \}$ \\

Build a derivation that proves the following: \\
$$\Gamma \cup \{ (\forall x)(\forall y)(P_3 (x,y) \leftrightarrow P_2 (x,y)) \} \vdash (\exists x)(\exists y)(\neg P_3 (x, y))$$
    }%
}
\newline
\newline\newline
\noindent\fbox{%
    \parbox{\textwidth}{%
First of all build the derivation without canceling any hypotheses:\\
    }%
}
\begin{scprooftree}{0.8}
  \AxiomC{$(\exists x)(\exists y)(\neg P_2 (x, y))$}
  \AxiomC{$(\exists y)(\neg P_2 (x, y))$}
  \AxiomC{$\neg P_2 (x, y)$}
  \AxiomC{$(\forall x)(\forall y)(P_3 (x,y) \leftrightarrow P_2 (x,y))$}
  \RightLabel{$E_{\forall *6}$}
  \UnaryInfC{$(\forall y)(P_3 (x,y) \leftrightarrow P_2 (x,y))$}
  \RightLabel{$E_{\forall *5}$}
  \UnaryInfC{$P_3 (x,y) \leftrightarrow P_2 (x,y)$}
  \AxiomC{$P_3 (x, y)$}
  \RightLabel{$E_{\leftrightarrow 1}$}
  \BinaryInfC{$P_2 (x, y)$}
  \RightLabel{$E_{\neg}$}
  \BinaryInfC{$\bot$}
  \LeftLabel{(3)}
  \RightLabel{$I_{\neg}$}
  \UnaryInfC{$\neg P_3 (x, y)$}
  \RightLabel{$I_{\exists *4}$}
  \UnaryInfC{$(\exists y)(\neg P_3 (x, y))$}
  \RightLabel{$I_{\exists *3}$}
  \UnaryInfC{$(\exists x)(\exists y)(\neg P_3 (x, y))$}
  \LeftLabel{(2)}
  \RightLabel{$E_{\exists *2}$}
  \BinaryInfC{$(\exists x)(\exists y)(\neg P_3 (x, y))$}
  \LeftLabel{(1)}
  \RightLabel{$E_{\exists *1}$}
  \BinaryInfC{$(\exists x)(\exists y)(\neg P_3 (x, y))$}
\end{scprooftree}


During the construction the following hypotheses were generated:
\begin{enumerate}
\item $(\exists y)(\neg P_2 (x, y))$
\item $\neg P_2 (x, y)$
\item $P_3 (x, y)$
\end{enumerate}

\noindent\fbox{%
    \parbox{\textwidth}{%
Next, start checking every atomic step top-down. In each atomic step, if the rule needs proofs we write them down. If the rule generated hypotheses we cancel them. If you can give a correct proof for all rules that need it, the derivation is correct. \\
    }%
}
\begin{itemize}
  \item [$E_{\forall *6}$]:
  \item This rule doesn't generate hypotheses.
  \item \textbf{*6 justification:} $x$ is free for $x$ in any formula (in particular, in: $(\forall y)(P_3 (x,y) \leftrightarrow P_2 (x,y))$).
\end{itemize}

\begin{itemize}
  \item [$E_{\forall *5}$:]
  \item This rule doesn't generate hypotheses.
  \item \textbf{*5 justification:} $y$ is free for $y$ in any formula (in particular, in $(P_3 (x,y) \leftrightarrow P_2 (x,y))$). 
\end{itemize}

\begin{itemize}
  \item [$E_{\leftrightarrow 1}$:]
  \item This rule doesn't generate hypotheses and doesn't need justification. 
\end{itemize}

\begin{itemize}
  \item [$E_{\neg 1}$:]
  \item This rule doesn't generate hypotheses doesn't need justification.
\end{itemize}
\newpage

\begin{itemize}
  \item [$I_{\neg 1}$:]
  \item This rule generates the hypotheses ($P_3 (x, y)$), se let's cancel it.
  \item This rule doesn't need justification.\\
\end{itemize}

The derivation at this step is: \\

\begin{scprooftree}{0.8}
  \AxiomC{$(\exists x)(\exists y)(\neg P_2 (x, y))$}
  \AxiomC{$(\exists y)(\neg P_2 (x, y))$}
  \AxiomC{$\neg P_2 (x, y)$}
  \AxiomC{$(\forall x)(\forall y)(P_3 (x,y) \leftrightarrow P_2 (x,y))$}
  \RightLabel{$E_{\forall *6}$}
  \UnaryInfC{$(\forall y)(P_3 (x,y) \leftrightarrow P_2 (x,y))$}
  \RightLabel{$E_{\forall *5}$}
  \UnaryInfC{$P_3 (x,y) \leftrightarrow P_2 (x,y)$}
  \AxiomC{$[P_3 (x, y)]^3$}
  \RightLabel{$E_{\leftrightarrow 1}$}
  \BinaryInfC{$P_2 (x, y)$}
  \RightLabel{$E_{\neg}$}
  \BinaryInfC{$\bot$}
  \LeftLabel{(3)}
  \RightLabel{$I_{\neg}$}
  \UnaryInfC{$\neg P_3 (x, y)$}
  \RightLabel{$I_{\exists *4}$}
  \UnaryInfC{$(\exists y)(\neg P_3 (x, y))$}
  \RightLabel{$I_{\exists *3}$}
  \UnaryInfC{$(\exists x)(\exists y)(\neg P_3 (x, y))$}
  \LeftLabel{(2)}
  \RightLabel{$E_{\exists *2}$}
  \BinaryInfC{$(\exists x)(\exists y)(\neg P_3 (x, y))$}
  \LeftLabel{(1)}
  \RightLabel{$E_{\exists *1}$}
  \BinaryInfC{$(\exists x)(\exists y)(\neg P_3 (x, y))$}
\end{scprooftree}

\begin{itemize}
  \item [$I_{\exists *4}$:]
  \item This rule doesn't generate hypotheses.
  \item \textbf{*4 justification:} $y$ is free for $y $ in any formula (in particular, in $\neg P_3 (x,y)$).
\end{itemize}

\begin{itemize}
  \item [$I_{\exists *3}$:]
  \item This rule doesn't generate hypotheses.
  \item \textbf{*3 justification:} $x$ is free for $x$ in any formula (in particular, in $(\exists y)(\neg P_3 (x,y))$).
\end{itemize}

\begin{itemize}
  \item [$E_{\exists *2}$:]
  \item This rule generates the hypotheses $\neg P_2 (x, y)$, let's cancel it.
  \item \textbf{*2 justification:} $y \not\in$ $FV((\exists x)(\exists y)(\neg P_3 (x, y)))$, $y \not\in$ $FV((\forall x)(\forall y)(P_3 (x,y) \leftrightarrow P_2 (x,y)))$. \\
  \textbf{Note that there's no need to check if other hypotheses contain $y$ free: ($P_3 (x,y)$ ni en $\neg P_2 (x,y)$) as in this step they are canceled.}\\
\end{itemize}

The derivation at this step is: \\
\begin{scprooftree}{0.8}
  \AxiomC{$(\exists x)(\exists y)(\neg P_2 (x, y))$}
  \AxiomC{$(\exists y)(\neg P_2 (x, y))$}
  \AxiomC{$[\neg P_2 (x, y)]^2$}
  \AxiomC{$(\forall x)(\forall y)(P_3 (x,y) \leftrightarrow P_2 (x,y))$}
  \RightLabel{$E_{\forall *6}$}
  \UnaryInfC{$(\forall y)(P_3 (x,y) \leftrightarrow P_2 (x,y))$}
  \RightLabel{$E_{\forall *5}$}
  \UnaryInfC{$P_3 (x,y) \leftrightarrow P_2 (x,y)$}
  \AxiomC{$[P_3 (x, y)]^3$}
  \RightLabel{$E_{\leftrightarrow 1}$}
  \BinaryInfC{$P_2 (x, y)$}
  \RightLabel{$E_{\neg}$}
  \BinaryInfC{$\bot$}
  \LeftLabel{(3)}
  \RightLabel{$I_{\neg}$}
  \UnaryInfC{$\neg P_3 (x, y)$}
  \RightLabel{$I_{\exists *4}$}
  \UnaryInfC{$(\exists y)(\neg P_3 (x, y))$}
  \RightLabel{$I_{\exists *3}$}
  \UnaryInfC{$(\exists x)(\exists y)(\neg P_3 (x, y))$}
  \LeftLabel{(2)}
  \RightLabel{$E_{\exists *2}$}
  \BinaryInfC{$(\exists x)(\exists y)(\neg P_3 (x, y))$}
  \LeftLabel{(1)}
  \RightLabel{$E_{\exists *1}$}
  \BinaryInfC{$(\exists x)(\exists y)(\neg P_3 (x, y))$}
\end{scprooftree}

\newpage

\begin{itemize}
\item [$E_{\exists *1}$:]
\item This rule generates the hypotheses: $(\exists y)(\neg P_2 (x, y))$, let's cancel it.
\item \textbf{*1 justification:} $x \not\in$ $FV((\exists x)(\exists y)(\neg P_3 (x, y)))$, $x \not\in$ $FV((\forall x)(\forall y)(P_3 (x,y) \leftrightarrow P_2 (x,y)))$.\\
\textbf{Let's note that for this case also there's no need to check if any other hypotheses contain $x$ free: ($P_3 (x,y)$ ni en $\neg P_2 (x,y)$) because they are all canceled.}\\
\end{itemize}

The derivation at this step is: \\

\begin{scprooftree}{0.8}
  \AxiomC{$(\exists x)(\exists y)(\neg P_2 (x, y))$}
  \AxiomC{$[(\exists y)(\neg P_2 (x, y))]^1$}
  \AxiomC{$[\neg P_2 (x, y)]]^2$}
  \AxiomC{$(\forall x)(\forall y)(P_3 (x,y) \leftrightarrow P_2 (x,y))$}
  \RightLabel{$E_{\forall *6}$}
  \UnaryInfC{$(\forall y)(P_3 (x,y) \leftrightarrow P_2 (x,y))$}
  \RightLabel{$E_{\forall *5}$}
  \UnaryInfC{$P_3 (x,y) \leftrightarrow P_2 (x,y)$}
  \AxiomC{$[P_3 (x, y)]^3$}
  \RightLabel{$E_{\leftrightarrow 1}$}
  \BinaryInfC{$P_2 (x, y)$}
  \RightLabel{$E_{\neg}$}
  \BinaryInfC{$\bot$}
  \LeftLabel{(3)}
  \RightLabel{$I_{\neg}$}
  \UnaryInfC{$\neg P_3 (x, y)$}
  \RightLabel{$I_{\exists *4}$}
  \UnaryInfC{$(\exists y)(\neg P_3 (x, y))$}
  \RightLabel{$I_{\exists *3}$}
  \UnaryInfC{$(\exists x)(\exists y)(\neg P_3 (x, y))$}
  \LeftLabel{(2)}
  \RightLabel{$E_{\exists *2}$}
  \BinaryInfC{$(\exists x)(\exists y)(\neg P_3 (x, y))$}
  \LeftLabel{(1)}
  \RightLabel{$E_{\exists *1}$}
  \BinaryInfC{$(\exists x)(\exists y)(\neg P_3 (x, y))$}
\end{scprooftree}

And this, together with the justifications is the complete finished derivation.

\newpage

\section{How to use identity rules.}

Special thanks to Prof. Luis Sierra (FIng UdelaR) for the corrections. \\

\noindent\fbox{%
    \parbox{\textwidth}{%
\textbf{From an university test:} \newline \newline
Build a derivation that proves:
$$(\forall x)P(x, f(x)), (\exists y)f(y) =' y \vdash (\exists z)P(f(z), z)$$
    }%
}
\newline \newline

First of all, looking at the premises we can figure out that to reach the conclusion, using substition in the application of the rule $E_{\forall}$ won't lead us anywhere. For example, if we wanted to reach $(\exists z)P(f(z), z)$ with the premise $(\forall x)P(x, f(x))$, we should apply the rule $E_{\forall}$. If we wanted to obtain $f(z)$ instead of $x$, when substituting, we would get $P(f(z), f(f(z)))$. The only way to swawp the arguments of $P$ is by using the rule $RI4'$. \\

Let's build the derivation: \\

All we can do with the premises we have is an existential elimination:

\begin{prooftree}
\AxiomC{$(\exists y)f(y) =' y$}
\AxiomC{$(\exists z)P(f(z), z)$}
\LeftLabel{(1)}
\RightLabel{$E_{\exists *1}$}
\BinaryInfC{$(\exists z)P(f(z), z)$}
\end{prooftree}

Here, we obtained the hypotheses $f(y) =' y$. Checking the premises again we can see that with $(\forall x)P(x, f(x))$, by applying the rule $E_{\forall}$ and substituting adequately we obtain $P(y, f(y))$. If we manage to substitute in this formula in a way that we get $P(f(y), y)$, applying the rule $I_{\exists}$ and substituting adequately we can obtain $(\exists z)P(f(z), z)$. We can do this by using the identity rule $RI4'$.
Doing this, the resulting derivation is:

\begin{prooftree}
  \AxiomC{$(\exists y)f(y) =' y$}
  \AxiomC{$(\forall x)P(f(x), x)$}
  \UnaryInfC{$P(f(y), y)$}
  \AxiomC{$f(y) =' y$}
  \AxiomC{$f(y) =' y$}
  \UnaryInfC{$y =' f(y)$}
  \TrinaryInfC{$P(f(y), y)$}
  \UnaryInfC{$(\exists z)P(f(z), z)$}
  \LeftLabel{(1)}
  \RightLabel{$E_{\exists *1}$}
  \BinaryInfC{$(\exists z)P(f(z), z)$}
\end{prooftree}

By adding the rules in each step and using the strategy of the previous example, we have built the derivation: \\

\noindent\fbox{%
    \parbox{\textwidth}{%
 \textbf{*4}: $y$ free for $x$ in $P(f(x), x).$ \\
 \textbf{*3}: $y$ free for $z_1$; $f(y)$ free for $z_2$ en $P(z_1,z_2)$. \\
 $f(y)$ free for $z_1$; $y$ free for $z_2$ en $P(z_1,z_2)$.\\
 \textbf{*2}: $y$ free for $z$ en $P(f(z), z)$.\\
 \textbf{*1}: $y$ $\not \in$ $FV(\{(\exists z)P(f(z), z), (\forall x)P(f(x), x)\}).$
    }%
}

\begin{prooftree}
  \AxiomC{$(\exists y)f(y) =' y$}
  \AxiomC{$[f(y) =' y]^1$}
  \RightLabel{$RI2$}
  \UnaryInfC{$y =' f(y)$}
  \AxiomC{$[f(y) =' y]^1$}
  \AxiomC{$(\forall x)P(x, f(x))$}
  \RightLabel{$E_{\forall *4}$}
  \UnaryInfC{$P(z_1,z_2)[y/z_1,f(y)/z_2])$}
  \RightLabel{$RI4'_{*3}$}
  \TrinaryInfC{$P(z_1,z_2)[f(y)/z_1,y/z_2)]$}
  \RightLabel{$I_{\exists *2}$}
  \UnaryInfC{$(\exists z)P(f(z), z)$}
  \LeftLabel{(1)}
  \RightLabel{$E_{\exists *1}$}
  \BinaryInfC{$(\exists z)P(f(z), z)$}
\end{prooftree}

\end{document}
